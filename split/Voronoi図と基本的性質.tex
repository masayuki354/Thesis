\documentclass[../main]{jsarticle}
\setcounter{section}{1}
\begin{document}
\subsection{ボロノイ図($Voronoi~diagram$)と基本的性質}
$2次元平面R^2における2点p,qに対して、そのユーグリッド距離をd(p,q)で表す。
平面上にn個の点の集合 S = \{p_1,p_2,...,p_n \}が与えられたとする。このとき$
\begin{equation}
	R(S;p_i) = \bigcap\limits_{p_j\in S \backslash \{p_i \} }\{p\in{\bf R}^2 | d(p,p_i) < d(p,p_j)\}
\end{equation}
 を$p_iの{\bf ボロノイ領域}(Voronoi~region)という。これは、平面上の点p\in{\bf R}^2で、Sの中で最も近い点がp_iであるという性質をもつものを集めてできる集合である。
これはSに属す各点p_iが、他よりも自分に近い点の集合を囲い込んでできる領域で、いわば勢力圏とみなすことができよう。\\$
$平面全体はR(S;p_1),R(S;p_2),R(S;p_3),...,R(S;p_n)とそれらの領域へ分解される。
この分割図形をSに対する{\bf ボロノイ図}(Voronoi~diagram)という。ボロノイ図の例を図1に示した。\\$
図中の水色の点がSの要素で実線がボロノイ領域の境界線である。

~ボロノイ図において、$2つのボロノイ領域の共通の境界を{\bf ボロノイ辺}(Voronoi~edge)と呼び、3つ以上のボロノイ領域の共通の境界点を{\bf ボロノイ点}(Voronoi~point)と呼ぶ
。Sの要素はこのボロノイ図の{\bf 生成元}(generator)または{\bf 母点}(generation~point)と呼ばれる。$\\
ボロノイ図の定義から導かれる基本的な性質をまとめておこう。

\end{document}