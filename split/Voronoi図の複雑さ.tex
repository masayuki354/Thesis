\documentclass[../main]{jsarticle}
\begin{document}
\subsection{Voronoi 図の複雑さ}
$次にボロノイ図の複雑さについてについてみてみよう。
母点の数をnとする : |S| = n.~ Sに対するボロノイ図のボロノイ辺の数をn_e,ボロノイ点の数をn_vとする。
また無限に伸びるボロノイ領域の数をn_1,有界なボロノイ領域の数をn_2とする。
ボロノイ領域の数と母点の数は一致するからn=n_1 + n_2 である。また、無限にのびるボロノイ辺の数はn_1に一致する。\\
~ここでSを囲む十分大きな閉曲線を考え、無限にのびるボロノイ辺は、この閉曲線上に端点を持つものとする。
こうすることでボロノイ辺やボロノイ点を見つけやすくなる。
無限にのびるボロノイ辺はn_1本あるから、この閉曲線上にはn_1個の端点ができ、閉曲線自体もn_1個の曲線分に分けられる。
そして、ボロノイ点および閉曲線上に設けた端点を{\bf 頂点}(vertex)と呼び、ボロノイ辺および閉曲線が分割されてできる曲線分を{\bf 辺}(edge)と呼ぶことにする。\\
~閉じたボロノイ図においては、各辺は端点を2つもち、各頂点は3本以上の辺と接続する。したがって$
\begin{equation}
2(n_e + n_1) \geq 3(n_v + n_1)
\end{equation}
$が成り立つ。特に、どのボロノイ点も退化していなければ式(1.2.1)は等号で成り立つ。\\~一方、平面上に描かれた連結な図形においては、\\頂点の数、辺の数、面の数をV,E,Fとするとオイラーの公式$
\begin{equation}
V - E + F = 2
\end{equation}
$が成り立つ。閉じたボロノイ図ではV = n_v+n_1,E=n_e+n_1である。また、ボロノイ領域に加えて閉曲線の外側にももう一つ面ができるからF=n+1である。これらを式(1.2.2)に代入して$
\begin{equation}
(n_v + n_1) - (n_e + n_1) + (n + 1) = 2
\end{equation}
$である。式(1.2.3)よりn_v = n_e - n + 1 であり、これを式(1.2.1)に代入して$
\begin{equation}
n_e \leq 3n - n_1 -3
\end{equation}
$を得る。\\
~一方、式(1.2.3)より n_e = n_v + n - 1 であるから、これを式(1.2.1)に代入して$
\begin{equation}
n_v \leq 2n - n_1 - 2
\end{equation}
$を得る。\\
よってn個の母点に対するボロノイ図のボロノイ辺の数n_eとボロノイ点の数n_vは式(1.2.4),(1.2.5)を満たす。\\
また、閉じたボロノイ図にはn+1個の面が含まれる。一方、それぞれの辺は両側の面の境界に貢献する。従って、すべての面の角数の合計は2(n_e+n_1)である。
1つの面当たりの平均の角数をfとすると、f = \frac{2(n_e+n_1)}{n+1}であるがこれは式(1.2.4)より$
\begin{equation}
f \leq \frac{2(3n - n_1 - 3 + n_1)}{n + 1} = 6\frac{n+1}{n+1} - \frac{12}{n+1} < 6 - \frac{12}{n+1}
\end{equation}
$となる。これより、ボロノイ領域をなす多角形の平均角数はほぼ6(6よりわずかに少ない)となる。\\
式(1.2.4),(1.2.5)より、ボロノイ辺の数は母点の3倍以下、ボロノイ点の数は母点の2倍以下であることが分かる。
従って、母点数をnとすると、ボロノイ辺の数もボロノイ点の数もO(n)である。このようにボロノイ図の構造はnに比例する複雑さしかない。
このことは、平面上のボロノイ図の顕著な性質である。この性質ゆえにボロノイ図は種々の幾何計算を効率よく行うために利用できる。$
\end{document}