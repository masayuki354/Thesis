\documentclass[../main]{jsarticle}
\begin{document}
\subsection{$Voronoi$図の逐次添加構成法}
ボロノイ図を構成するためのアルゴリズムにはいくつかあるがここでは逐次添加構成法を扱う。
この方法は最悪の場合の計算量は$O(n^2)であり、他の方法に劣るが、一般の母点分布に対する平均の計算量をO(n)に抑えることができ、実装に容易なので実用的と言われている。\\
{\bf 逐次添加構成法}とは少数のm個の母点P_i (i =1,...,m)からなるVoronoi図V_mに新しい母点P_{m+1}を添加してV_{m+1}を作る操作を繰り返すものである。\\
母点P_{m+1}を添加するには以下の操作が必要である。\\
(I)P_i (i=1,...,m)の中から新たな母点P_{m+1}に最も近い母点P_n を求める。\\
(I\hspace{-.1em}I)P_{m+1}のVoronoi多角形を求めV_mのうちP_{m+1}のVoronoi多角形の内部を消去する。\\$
母点を新たに一つ添加する様子を以下の図に示した。

$この図の(a)に黒点で示した母点に対するボロノイ図が実線のように得られているときに赤丸で示した新たな母点を添加したとしよう。
まず、ボロノイ点の中でこれまでの母点よりも新しい母点に近いものをすべて見つける。この例では青色の線に囲まれた3つの交点がボロノイ点である。
次に、これらの交点のボロノイ点を境界上にもつ母点と新しい母点との間の垂直二等分線によって、古いボロノイ領域を2つに分ける。
このとき、垂直二等分線は、交点のボロノイ点とそれ以外のボロノイ点をつなぐボロノイ辺の途中を通過する。したがって2つに分けた領域のうち新しい母点に近い方を集めると、一つの凸多角形となり、これが新しい母点のボロノイ領域となる。
そこで最後にこの内部のボロノイ点とボロノイ辺を除くと図3(b)に示すように、ボロノイ図の更新作業が完成する。\\
そして取り除かれるボロノイ点には次のような性質がある。$

\end{document}