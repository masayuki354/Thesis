\documentclass[../main]{jsarticle}
\begin{document}
$
では、この出発点となるPをうまく見つける方法はないだろうか。\\
~そこで母点すべてを含む正方形を考える。
この正方形を縦横の直線で四等分しそれぞれの正方形を同じように四等分するということをk回くり返す。
こうして生まれた最小の正方形を{\bf バケット}とよぶ。そしてこの四分割に対応してグラフを考える。
まず、すべての母点含む正方形をこのグラフの根に対応させる。
次にこの正方形を四分割してできる四つの小正方形をそれぞれ正方形の節点の子供の節点とみなして、枝の先にぶら下げる。
さらにそれぞれの小正方形の四分割に対応してそれぞれ4個の子供の節点を作ることを繰り返す。
これにより、根からk本の辺を下へたどると、バケットへたどり着く。\\
この四分木の各節点にひとつずつ母点番号を格納する場所を用意する。
この格納場所は最初は空である。
最初の一個を格納した時にその母点が属すバケットから根へ向かって辺をたどり、その時に通過するすべての節点にこの母点番号を格納していく。
そのあとは、新しい母点が添加されるたびに、それぞれが属すバケットから辺をたどり格納場所が空である限り、その母点番号を格納していく。\\
~V_mにP_{m+1}が添加されたとしよう。
P_{m+1}が属すバケットから根へ向かって辺をたどり空の格納場所に母点番号を格納していくがいずれ空でない節点に出会う。
そこに格納されている母点番号をもつ母点をPとする。
そしてそのPに対して出発点の探索を行うことでP_{m+1}に最も近い母点を見つけることができる。
いま、Pに対し出発点を探索しを行いP_jが得られたとしよう。このP_jは全ての母点の中でP_{m+1}に最も近い。
そこでP_jのボロノイ領域の境界上のボロノイ点の中で最もP_{m+1}に近いものをも見つける。
これをqとしよう。
q は、すべてのボロノイ点の中でP_{m+1}に最も近くP_{m+1}が作るボロノイ領域に含まれる。
すなわち、 q は取り除かれるべきボロノイ点の一つである。この q を出発点として取り除かれるべきボロノイ点を探索することができる。
ここで、すべての母点を含む正方形を分割する回数 k はパラメータ\alpha を用いてk = \lfloor {\log}_4{(\alpha n)} \rfloor ぐらいがよい。
すなわち、4^kがnに近くなるように選ぶとよい。
こうするとこで、各バケットに一つくらいの母点が含まれることになるので常にP_{m+1}に近い出発点を選ぶことができ、nが大きくなっても出発点の探索を行う回数がnに依らない定数程度で済むからである。$


\end{document}